\section{All Models Are Wrong (But Not All Forecasts)}

An infamous quote across modeling domains hypothesizes that "all models are wrong, but some are useful" \autocite{box2007response}. In this section I introduce the interplay between modeling domains of weather forecasts (meteorology), streamflow forecasts (hydrology), and groundwater availability prediction (hydrogeology). Each domain is sensitive to different physical laws and limitations on numerical solutions, but more germane to this conversation is how model verification could apply to the field.
% * <jeff@jnorville.com> 2016-09-08T07:22:10.204Z:
%
% In fact the original Box quote was from the first edition of the same book in 1985 -- but I assume I should reference the version in print. Thoughts?
%
% ^.

\section{Meteorology}

Predictive modeling in meteorology is generally conducted on a grid system which has relatively little to do with the Earth's surface. The reasoning is simple: complexity is in convection, the pressure gradients through stratified layers of atmosphere. The planet surface does indeed influence meteorology, but there is insufficient motivation to spatially adjust model grids to follow terrain.

Forecast ranges are generally divided into short-range (hours to days), medium-range (couple weeks), long range (months) and seasonal (three to four month blocks by season, typically DJF, MAM, JJA, SON).  \autocite{JolliffeIanT.andStephenson2012ForecastVerification}.

\section{Hydrology}

In contrast to the grid-based forecasts the meteorologists build, scientists concerned with surface water flow generally implement basin-based models, with point-based observations. Hydrological models are generally dominated by topography, then climate, soil type and vegetation. 

Downscaling climate and meteorological information to the terrain or basin level is a difficult science of its own. It is particularly applicable to our scoreboard, however, as we have focused this release on discharge scores: inherent in a score is a paired prediction and observation. Often the larger models (ex LISFLOOD by JRC) use what's considered reliable post-processed node flow in place of observation at an unobserved (or unmonitored) forecast point.

% * <jeff@jnorville.com> 2016-09-09T12:20:06.737Z:
%
% Jeff to add discussion from Crochemore, Arnal, Thirel, C Perrin seasonal forecasts
%
% ^.
\section{Hydrogeology}

There is a relatively narrow link (and therefor area of interest) between seasonal forecasts, hydrology, and groundwater, but considering my background (and my roots in central California) it's important to mention it here. 

The component of precipitation which infiltrates into the ground to become groundwater is called \gls{recharge}, and estimating the relationship between natural recharge and precipitation may be complicated by soil properties, frozen precipitation, thawing ground, evaporation and transpiration (by plants), and other processes \autocite{fetter1994applied}. Seasonal forecasts are increasingly included in groundwater studies and sustainability analyses for aquifer development, and to avoid over-extraction. %cite Porcello?

Seasonal forecasts have more uncertainty in predictions of precipitation than temperature; seasonal forecasts along California's Sierra Nevada mountains suggest fewer changes in precipitation volume but significant changes in spring low temperatures.

Higher snow water equivalent and earlier rainfall and melt cycles result in earlier, higher runoff and flooding; smaller-volume snow pack: and ultimately earlier desiccation of the low-elevation Sierra Nevada, exacerbated fire seasons, and ultimately less groundwater recharge. \autocite{guan20132010} 

% Annual recharge rates for groundwater models abstracted from meteorology forecasts / seasonal forecasts - so many examples


% (particularly focusing on atmospheric river phenomenon) 

% http://www.env.gov.bc.ca/wsd/data_searches/obswell/obsw002.html


% disappearing lakes
% http://www.ecmwf.int/en/about/media-centre/news/2016/earth-system-modelling-takes-centre-stage-annual-seminar
